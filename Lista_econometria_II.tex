\documentclass[]{article}
\usepackage{lmodern}
\usepackage{amssymb,amsmath}
\usepackage{ifxetex,ifluatex}
\usepackage{fixltx2e} % provides \textsubscript
\ifnum 0\ifxetex 1\fi\ifluatex 1\fi=0 % if pdftex
  \usepackage[T1]{fontenc}
  \usepackage[utf8]{inputenc}
\else % if luatex or xelatex
  \ifxetex
    \usepackage{mathspec}
  \else
    \usepackage{fontspec}
  \fi
  \defaultfontfeatures{Ligatures=TeX,Scale=MatchLowercase}
\fi
% use upquote if available, for straight quotes in verbatim environments
\IfFileExists{upquote.sty}{\usepackage{upquote}}{}
% use microtype if available
\IfFileExists{microtype.sty}{%
\usepackage{microtype}
\UseMicrotypeSet[protrusion]{basicmath} % disable protrusion for tt fonts
}{}
\usepackage[margin=1in]{geometry}
\usepackage{hyperref}
\hypersetup{unicode=true,
            pdftitle={Lista para Modelo Dinâmico},
            pdfauthor={Thiago Mendes Rosa},
            pdfborder={0 0 0},
            breaklinks=true}
\urlstyle{same}  % don't use monospace font for urls
\usepackage{color}
\usepackage{fancyvrb}
\newcommand{\VerbBar}{|}
\newcommand{\VERB}{\Verb[commandchars=\\\{\}]}
\DefineVerbatimEnvironment{Highlighting}{Verbatim}{commandchars=\\\{\}}
% Add ',fontsize=\small' for more characters per line
\usepackage{framed}
\definecolor{shadecolor}{RGB}{248,248,248}
\newenvironment{Shaded}{\begin{snugshade}}{\end{snugshade}}
\newcommand{\AlertTok}[1]{\textcolor[rgb]{0.94,0.16,0.16}{#1}}
\newcommand{\AnnotationTok}[1]{\textcolor[rgb]{0.56,0.35,0.01}{\textbf{\textit{#1}}}}
\newcommand{\AttributeTok}[1]{\textcolor[rgb]{0.77,0.63,0.00}{#1}}
\newcommand{\BaseNTok}[1]{\textcolor[rgb]{0.00,0.00,0.81}{#1}}
\newcommand{\BuiltInTok}[1]{#1}
\newcommand{\CharTok}[1]{\textcolor[rgb]{0.31,0.60,0.02}{#1}}
\newcommand{\CommentTok}[1]{\textcolor[rgb]{0.56,0.35,0.01}{\textit{#1}}}
\newcommand{\CommentVarTok}[1]{\textcolor[rgb]{0.56,0.35,0.01}{\textbf{\textit{#1}}}}
\newcommand{\ConstantTok}[1]{\textcolor[rgb]{0.00,0.00,0.00}{#1}}
\newcommand{\ControlFlowTok}[1]{\textcolor[rgb]{0.13,0.29,0.53}{\textbf{#1}}}
\newcommand{\DataTypeTok}[1]{\textcolor[rgb]{0.13,0.29,0.53}{#1}}
\newcommand{\DecValTok}[1]{\textcolor[rgb]{0.00,0.00,0.81}{#1}}
\newcommand{\DocumentationTok}[1]{\textcolor[rgb]{0.56,0.35,0.01}{\textbf{\textit{#1}}}}
\newcommand{\ErrorTok}[1]{\textcolor[rgb]{0.64,0.00,0.00}{\textbf{#1}}}
\newcommand{\ExtensionTok}[1]{#1}
\newcommand{\FloatTok}[1]{\textcolor[rgb]{0.00,0.00,0.81}{#1}}
\newcommand{\FunctionTok}[1]{\textcolor[rgb]{0.00,0.00,0.00}{#1}}
\newcommand{\ImportTok}[1]{#1}
\newcommand{\InformationTok}[1]{\textcolor[rgb]{0.56,0.35,0.01}{\textbf{\textit{#1}}}}
\newcommand{\KeywordTok}[1]{\textcolor[rgb]{0.13,0.29,0.53}{\textbf{#1}}}
\newcommand{\NormalTok}[1]{#1}
\newcommand{\OperatorTok}[1]{\textcolor[rgb]{0.81,0.36,0.00}{\textbf{#1}}}
\newcommand{\OtherTok}[1]{\textcolor[rgb]{0.56,0.35,0.01}{#1}}
\newcommand{\PreprocessorTok}[1]{\textcolor[rgb]{0.56,0.35,0.01}{\textit{#1}}}
\newcommand{\RegionMarkerTok}[1]{#1}
\newcommand{\SpecialCharTok}[1]{\textcolor[rgb]{0.00,0.00,0.00}{#1}}
\newcommand{\SpecialStringTok}[1]{\textcolor[rgb]{0.31,0.60,0.02}{#1}}
\newcommand{\StringTok}[1]{\textcolor[rgb]{0.31,0.60,0.02}{#1}}
\newcommand{\VariableTok}[1]{\textcolor[rgb]{0.00,0.00,0.00}{#1}}
\newcommand{\VerbatimStringTok}[1]{\textcolor[rgb]{0.31,0.60,0.02}{#1}}
\newcommand{\WarningTok}[1]{\textcolor[rgb]{0.56,0.35,0.01}{\textbf{\textit{#1}}}}
\usepackage{graphicx,grffile}
\makeatletter
\def\maxwidth{\ifdim\Gin@nat@width>\linewidth\linewidth\else\Gin@nat@width\fi}
\def\maxheight{\ifdim\Gin@nat@height>\textheight\textheight\else\Gin@nat@height\fi}
\makeatother
% Scale images if necessary, so that they will not overflow the page
% margins by default, and it is still possible to overwrite the defaults
% using explicit options in \includegraphics[width, height, ...]{}
\setkeys{Gin}{width=\maxwidth,height=\maxheight,keepaspectratio}
\IfFileExists{parskip.sty}{%
\usepackage{parskip}
}{% else
\setlength{\parindent}{0pt}
\setlength{\parskip}{6pt plus 2pt minus 1pt}
}
\setlength{\emergencystretch}{3em}  % prevent overfull lines
\providecommand{\tightlist}{%
  \setlength{\itemsep}{0pt}\setlength{\parskip}{0pt}}
\setcounter{secnumdepth}{0}
% Redefines (sub)paragraphs to behave more like sections
\ifx\paragraph\undefined\else
\let\oldparagraph\paragraph
\renewcommand{\paragraph}[1]{\oldparagraph{#1}\mbox{}}
\fi
\ifx\subparagraph\undefined\else
\let\oldsubparagraph\subparagraph
\renewcommand{\subparagraph}[1]{\oldsubparagraph{#1}\mbox{}}
\fi

%%% Use protect on footnotes to avoid problems with footnotes in titles
\let\rmarkdownfootnote\footnote%
\def\footnote{\protect\rmarkdownfootnote}

%%% Change title format to be more compact
\usepackage{titling}

% Create subtitle command for use in maketitle
\providecommand{\subtitle}[1]{
  \posttitle{
    \begin{center}\large#1\end{center}
    }
}

\setlength{\droptitle}{-2em}

  \title{Lista para Modelo Dinâmico}
    \pretitle{\vspace{\droptitle}\centering\huge}
  \posttitle{\par}
    \author{Thiago Mendes Rosa}
    \preauthor{\centering\large\emph}
  \postauthor{\par}
      \predate{\centering\large\emph}
  \postdate{\par}
    \date{13/06/2019}


\begin{document}
\maketitle

\hypertarget{teoria}{%
\section{Teoria}\label{teoria}}

\hypertarget{modelo-economico}{%
\subsection{Modelo Econômico}\label{modelo-economico}}

Harold Zurcher gerencia uma frota de ônibus que é sujeita a todo tipo de
problema quando esta na rua. A milhagem (quilometragem) acumulada de um
ônibus \(x_t\) é a variável de estado do problema. O desgaste do ônibus
afeta o custo operacional esperado \(c(x_t; \theta_1)\) que depende da
milhagem e um vetor de parâmetros não conhecido
\(\theta_1 = \{\theta_{11}, . . . , \theta_{1n}\}\).

Assuma que os custos dos ônbius vem de dois componentes: manutenção
regular e despesas operacionais \(m(.)\) e o custo \(f(.)\) de
substituir o motor no caso de falha (que é um evento estocástico que
ocorre com alguma probabilidade).

\textbf{a) Escreva o custo como uma combinação destes dois componentes.
Indique claramente quais elementos são função de \(x_t\). Argumente
informalmente que \(\frac{\partial c}{\partial x_t} > 0\). Esta hipótese
é necessária para a solução do modelo. Ela é uma boa hipótese? Você pode
fazer um argumento para explicar por que
\(\frac{\partial c}{\partial x_t}\) poderia ser negativa (pelo menos
para alguns valores de \(x_t\))?}

\emph{Considere que o custo \(m(.)\) pode ser decomposto em
\(m(m_r(x_t),m_o(.))\), em que \(m_r\) denota as desposas com manutenção
regular e \(m_o\) as despesas operacionais. A função custo pode ser
escrita como \(c=(m(m_r(x_t),m_o(.)),f(x_t,.))\). Espera-se que, quanto
maior for a utilização de um ônibus da frota, maiores serão tanto os
custos de manutenção regular quanto os custos para substituição do motor
em caso de falha. Com isso, quanto mais o ônibus roda, maior é o
desgaste de suas peças e, portanto, maiores serão os custos. O custo
poderia diminuir com \(x_t\) em seus valores iniciais se, por exemplo,
houvesse uma garantia do fornecedor para qualquer tipo de problema nas
milhas iniciais.}

\begin{Shaded}
\begin{Highlighting}[]
\CommentTok{# Carregar a base de dados}

\CommentTok{#### Ler o dicionário de variáveis}
\CommentTok{#  Extrair tabela do PDF com a descrição das variáveis}
\NormalTok{dic <-}\StringTok{ }\NormalTok{tabulizer}\OperatorTok{::}\KeywordTok{extract_tables}\NormalTok{(}\StringTok{"lista2019.pdf"}\NormalTok{,}
                                 \DataTypeTok{output =} \StringTok{"data.frame"}\NormalTok{)[[}\DecValTok{1}\NormalTok{]]}

\CommentTok{# Listar arquivos com as bases de dados}
\NormalTok{bases<-}\StringTok{ }\KeywordTok{data.frame}\NormalTok{(}\DataTypeTok{arquivo=}\KeywordTok{list.files}\NormalTok{(}\StringTok{"rust_data"}\NormalTok{),}
                   \DataTypeTok{nome=}\KeywordTok{gsub}\NormalTok{(}\StringTok{".asc"}\NormalTok{,}\StringTok{""}\NormalTok{, }\KeywordTok{list.files}\NormalTok{(}\StringTok{"rust_data"}\NormalTok{)),}
                   \DataTypeTok{linhas=}\KeywordTok{c}\NormalTok{(}\KeywordTok{rep}\NormalTok{(}\DecValTok{137}\NormalTok{,}\DecValTok{4}\NormalTok{),}\DecValTok{128}\NormalTok{,}\DecValTok{36}\NormalTok{,}\DecValTok{60}\NormalTok{,}\DecValTok{81}\NormalTok{),}
                   \DataTypeTok{stringsAsFactors =}\NormalTok{ F)}

\CommentTok{# Looping para carregar as bases}
\ControlFlowTok{for}\NormalTok{(b }\ControlFlowTok{in}\NormalTok{ bases}\OperatorTok{$}\NormalTok{arquivo)\{}

  \CommentTok{# Ler a base de dados}
\NormalTok{  base <-}\StringTok{ }\NormalTok{data.table}\OperatorTok{::}\KeywordTok{fread}\NormalTok{(}\KeywordTok{paste0}\NormalTok{(}\StringTok{"rust_data/"}\NormalTok{,b))}
  
  \CommentTok{# Definir o número de linhas da matriz}
\NormalTok{  nl<-bases[bases}\OperatorTok{$}\NormalTok{arquivo}\OperatorTok{==}\NormalTok{b,]}\OperatorTok{$}\NormalTok{linhas}
  
  \CommentTok{# Definir número de meses existentes na base}
\NormalTok{  nm<-nl}\DecValTok{-12}\OperatorTok{+}\DecValTok{1}
  
  \CommentTok{# Criar objeto para receber os dados fixos de cada ônibus}
\NormalTok{  dados <-}\StringTok{ }\KeywordTok{c}\NormalTok{()}
  
  \CommentTok{# Criar um objeto para receber as informações mensais}
\NormalTok{  t<-}\KeywordTok{c}\NormalTok{()}

\CommentTok{# Criar um objeto para receber as referências}
\NormalTok{  ref<-}\KeywordTok{c}\NormalTok{()}

\CommentTok{# Iniciar looping para carregar as informações fixas,}
\CommentTok{# repetindo para o número de meses}
\ControlFlowTok{for}\NormalTok{(i }\ControlFlowTok{in} \DecValTok{1}\OperatorTok{:}\DecValTok{11}\NormalTok{)\{}
  
  \CommentTok{# Capturar as informações de cada ônibus}
  \KeywordTok{assign}\NormalTok{(}\KeywordTok{paste0}\NormalTok{(}\StringTok{"V"}\NormalTok{,i), }\KeywordTok{unlist}\NormalTok{(}\KeywordTok{rep}\NormalTok{(base[}\KeywordTok{seq}\NormalTok{(i,}\KeywordTok{nrow}\NormalTok{(base),nl),],nm)))}
  
  \CommentTok{# Juntar resultados}
\NormalTok{  dados <-}\StringTok{ }\KeywordTok{cbind}\NormalTok{(dados,}\KeywordTok{get}\NormalTok{(}\KeywordTok{paste0}\NormalTok{(}\StringTok{"V"}\NormalTok{,i)))}
  
\NormalTok{\}}

\CommentTok{# Retirar as informações mensais (odômetros)}
  
  \ControlFlowTok{for}\NormalTok{(j }\ControlFlowTok{in} \DecValTok{12}\OperatorTok{:}\NormalTok{nl)\{}
    
    
    \CommentTok{# Retirar para cada ônibus}
\NormalTok{    V12 <-}\StringTok{ }\NormalTok{base[}\KeywordTok{seq}\NormalTok{(j,}\KeywordTok{nrow}\NormalTok{(base),nl)]}
    
    \CommentTok{# Juntar resultados}
\NormalTok{    t <-}\StringTok{ }\KeywordTok{rbind}\NormalTok{(t,V12)}
\NormalTok{\}}

\CommentTok{# Criar um objeto com a referência,}
\CommentTok{# tendo como base o mês e ano inicial do odômetro}

  
  \ControlFlowTok{for}\NormalTok{(i }\ControlFlowTok{in} \KeywordTok{seq}\NormalTok{(}\DecValTok{10}\NormalTok{,}\KeywordTok{nrow}\NormalTok{(base),nl))\{}
    
    \CommentTok{# Capturar o mês inicial, o ano inicial, definir sempre o primeiro}
    \CommentTok{# dia de cada mês para transformar em data e criar a sequência de }
    \CommentTok{# meses conforme o número de meses(nm) disponíveis na base}
    
\NormalTok{    r <-}\StringTok{ }\KeywordTok{data.frame}\NormalTok{(}\DataTypeTok{ref=}\KeywordTok{seq}\NormalTok{(}
      \CommentTok{# Define a data}
\NormalTok{      lubridate}\OperatorTok{::}\KeywordTok{dmy}\NormalTok{(}\KeywordTok{paste}\NormalTok{(}\StringTok{"01"}\NormalTok{,base[i],}
\NormalTok{                           base[i}\OperatorTok{+}\DecValTok{1}\NormalTok{],}\DataTypeTok{sep =} \StringTok{"/"}\NormalTok{)),}
      \CommentTok{# Sequência mensal}
      \DataTypeTok{by=}\StringTok{"month"}\NormalTok{,}
      \CommentTok{# Pelo número de meses}
      \DataTypeTok{length.out =}\NormalTok{ nm))}
  
    \CommentTok{# Juntar resultados}
\NormalTok{    ref<-}\StringTok{ }\KeywordTok{rbind}\NormalTok{(ref,r)}
  
\NormalTok{\}}

\CommentTok{# Juntar a base final}
\KeywordTok{assign}\NormalTok{(bases[bases}\OperatorTok{$}\NormalTok{arquivo}\OperatorTok{==}\NormalTok{b,]}\OperatorTok{$}\NormalTok{nome,}
\KeywordTok{cbind}\NormalTok{(dados,}\DataTypeTok{V12=}\NormalTok{t) }\OperatorTok
\StringTok{  }\CommentTok{# Organizar por ônibus}
\StringTok{  }\NormalTok{dplyr}\OperatorTok{::}\KeywordTok{arrange}\NormalTok{(V1) }\OperatorTok
\StringTok{  }\CommentTok{# Trazer referência}
\StringTok{  }\NormalTok{dplyr}\OperatorTok{::}\KeywordTok{bind_cols}\NormalTok{(ref) }\OperatorTok
\StringTok{  }\CommentTok{# Ajustar odômetros para ocasião da troca}
\StringTok{  }\NormalTok{dplyr}\OperatorTok{::}\KeywordTok{mutate}\NormalTok{(}\DataTypeTok{V12_adj=}\KeywordTok{case_when}\NormalTok{(V12.V1}\OperatorTok{>}\NormalTok{V6}\OperatorTok{&}\NormalTok{V6}\OperatorTok{>}\DecValTok{0}\OperatorTok{~}\NormalTok{V12.V1}\OperatorTok{-}\NormalTok{V6,}
                                \OtherTok{TRUE}\OperatorTok{~}\NormalTok{V12.V1), }\CommentTok{# Primeira troca}
                \DataTypeTok{V12_adj=}\KeywordTok{case_when}\NormalTok{(V12.V1}\OperatorTok{>}\NormalTok{V9}\OperatorTok{&}\NormalTok{V9}\OperatorTok{>}\DecValTok{0}\OperatorTok{~}\NormalTok{V12.V1}\OperatorTok{-}\NormalTok{V9,}
                                \OtherTok{TRUE}\OperatorTok{~}\NormalTok{V12_adj))) }\CommentTok{# Segunda troca}

\CommentTok{# Remover objetos desnecessários}
\KeywordTok{rm}\NormalTok{(base,dados,r,ref,t,}\DataTypeTok{list=}\KeywordTok{ls}\NormalTok{(}\DataTypeTok{pattern =} \StringTok{"V}\CharTok{\textbackslash{}\textbackslash{}}\StringTok{d"}\NormalTok{),b,i,j,nl,nm)}
\NormalTok{\}}
\end{Highlighting}
\end{Shaded}


\end{document}
